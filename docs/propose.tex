\documentclass{article}
\usepackage[final]{nips_2017}
\usepackage[utf8]{inputenc} % allow utf-8 input
\usepackage[T1]{fontenc}    % use 8-bit T1 fonts
\usepackage{hyperref}       % hyperlinks
\usepackage{url}            % simple URL typesetting
\usepackage{booktabs}       % professional-quality tables
\usepackage{amsfonts}       % blackboard math symbols
\usepackage{nicefrac}       % compact symbols for 1/2, etc.
\usepackage{microtype}      % microtypography
\usepackage{graphicx}
\usepackage{wrapfig}
\title{Family kinship Recognition Using Deep Learning}

\author{
  Bruce Jianye Liu\\
  Department of Computer Science\\
  Stanford University\\
  \texttt{bruceliu@stanford.edu} \\
}

\begin{document}
% \nipsfinalcopy is no longer used

\begin{center}
\includegraphics[width=3cm, height=0.7cm]{CS230}
\end{center}

\maketitle

\section{Problem Description}
Half of genetic information is passed down from parents to children. Therefore
people biologically related share delicate similiarities. This declicacy could
be caught by human eyes, by looking at family photos. While computer vision
performance improving in the decade, it becomes impossible to use deep learning
model to capture the different. Kinship recognition could lead variety of
usefull applications in reality such as missing-children and parents matching,
family album organization, socal networking apps, lost sibling/relatives
searching, crime investigation. In this paper, we propose a fine-tuned FaceNet
model to identify the relationship between two faces -- parent-children, sibling-sibling, or none-kinship.

\section{Related Works}

It might be hard to achieve higher accuracy since people without any family
kinship would look simliar to each other. There are other metrics that we could
consider to determine if the kinship exists. Such information includes hands
shape, nail types, foot toes, ears, hairs. Due to time limitation and team
size, we aren't able to collect this data. Otherwise the prediction would
improve a lot.

Kin recognition work mainly focus on kinship verification, to decide whether
two faces have kinship, family classification, classifing a face to a family,
and family member regcontion, determining 2 faces siblings or parents. Over the
past decades, many methods has been proposed on this research, including
hand-crafted feature, face ecnodings, and metric learning[3].

\section{Dataset}

\begin{wrapfigure}{r}{0.25\textwidth}
\includegraphics[width=0.9\linewidth]{facepairs}
\caption{FIW dataset}
\label{fig:wrapfig}
\end{wrapfigure}

We are going to use Families In The Wild (FIW) Database[1]. FIW is the largest and
most comprehensive database available for kinship recognition. We use version 0.1.2 at writing time, which include 13,188 faces from 1018 families. Event though it has 11 kinship types, father-daughter (F-D), father-son (F-S), mother-daughter (M-D), mother-son (M-S), brother-brother (B-B), sister-sister (S-S), grandfather-granddaughter (GF-GD), grandfather-grandson (GF-GS),  grandmother-granddaughter (GM-GD), grandmother-grandson (GM-GS), to our study, siblings and parent-child types are what we are going to use. Adding up 64669 F-D, 46143 F-S, 68935 M-D, 48940 M-S types, we get 22687 parent-child photos while sibling types contains 55937 photos. All face images are 108*124*3 size.

There are another smaller family dataset KinFaceW-I and KinFaceW-II, which
includes 533 pairs of parent-child type images.

We divided 95 of the images as training set, 5 percent as test set.

\section{Models and Evaluation}
we use a multi-classifier to indicate what the kinship between two faces is. The
output layer we choose is softmax layer and label could be parent-children,
sibling-sibling or none-kinship.  Triplets training would be used.

FaceNet[2] performs very well in face recognition tasks. We are going to tweak the
model little bits to train our model.

\section*{References}
\medskip
\small
[1] Robinson, Joseph P., et al. Families in the Wild (FIW): Large-Scale Kinship
Image Database and Benchmarks. {\it Proceedings of the 2016 ACM on Multimedia
Conference} - MM '16, 2016, doi:10.1145/2964284.2967219.

[2] Schroff, F., Kalenichenko, D., \& Philbin, J. (2015). FaceNet: A unified
embedding for face recognition and clustering. {\it 2015 IEEE Conference on
Computer Vision and Pattern Recognition (CVPR)}. doi: 10.1109/cvpr.2015.7298682

[3] Wang, S., Robinson, J. P., & Fu, Y. (2017, May). Kinship verification on
families in the wild with marginalized denoising metric learning. In 2017 12th
IEEE International Conference on Automatic Face & Gesture Recognition (FG 2017)
(pp. 216-221). IEEE.

\end{document}
